\documentclass[12pt]{article}

\usepackage{fancyhdr}
\usepackage{amsthm}
\usepackage{amsmath}
\usepackage{graphicx}
\usepackage{amssymb}
\usepackage{esint}
\usepackage{subfigure}
\usepackage{color}
\usepackage{moreverb}
\usepackage{wrapfig}

\textwidth 17cm \topmargin -1cm \oddsidemargin 0cm \textheight 21.5cm
\pagestyle{empty} \pagestyle{fancyplain}
\lhead[\fancyplain{}{}]{\fancyplain{}{{\sc Adam Farnsworth, Myles Adams:}}}
\chead[\fancyplain{}{}]{\fancyplain{}{{\sc Hw 1}}}
\rhead[\fancyplain{}{}]{\fancyplain{}{{\sc Fall 2017}}}

\newcommand{\etal}{\textit{et al. }}

\begin{document}
\centerline{\Large\textbf{Question 1}}
\vspace{2cm}

\section{Introduction}\label{sec::Intro}
The goal of this project is to implement three different numerical methods for solving a general ODE and study their accuracy.  These numerical methods are as follows:
\begin{itemize}
\item The Euler method.
\item The explicit trapezoidal method (RK2).
\item The classical Runge-Kutta method (RK4).
\end{itemize}
We will use these to study a falling object under the force of gravity and drag force.  This will be seen by comparing the numerical solution and the exact solution graphically.  

\section{Shared equations and values}\label{sec::equations and values}
All three methods will use the ODE bellow:
$$
\left\{ \begin{array}{rl}
\frac{dy}{dt} = f(t, y), \mbox{ } \\
y(t_0) = y_0 \mbox{ }
       \end{array} \right.
$$
where $y = y(t)$ is the unknown solution we seek to approximate, $f = f(t, y)$ is a given function,
$t_0$ is the given initial time and $y_0$ the given initial condition.  In order to check the accuracy, we consider the following system describing a falling object under the force of gravity and drag force:
$$
\left\{ \begin{array}{rl}
\frac{dv}{dt}  = g -\frac{c_d}{m}v^2, \mbox{ } \\
v(0) = 0 \mbox{ }
       \end{array} \right.
$$
where $g \approx 9.81\frac{m}{s^2}$ is the free-fall acceleration, $m = 75 kg$ is the mass of the object and
$c_d = 0.25 \frac{kg}{m}$ is the drag coefficient. In this case $f(t, v) = g −\frac{c_d}{m}v^2$ and the exact solution is:

\begin{equation}
v(t)=\sqrt{\frac{gm}{c_d}}tanh(t\sqrt{\frac{gc_d}{m}})\nonumber
\end{equation}

and we will take $t_f = 15s$ and a time step of $\Delta t = 0.1s, 0.05s, 0.025s, 0.0125s, 3.3 s$

\section{Methods}\label{sec::Euler, RK2, RK4}

\subsection{Euler}\label{sec::euler}
%put stuff here for Euler

\subsection{Explicit trapezoidal}\label{sec::RK2}
%put stuff here for RK2

\subsection{Classical Runge-Kutta}\label{sec::RK4}
%put stuff here for RK4









%%%%%%%%%%%
\newpage
\clearpage
\centerline{\Large\textbf{Question 2}}
\section{Introduction}\label{sec::Intro}
Write intro here for question 2


%%%%%%%%%%%


%%%%%%%%%%%
\newpage
\clearpage
\setcounter{page}{1} \pagestyle{empty}
\section{References}\label{sec::References}
[1] Daniil Bochkov, CS 111 - Introduction to Computational Science Homework 1

    Fall 2017

%%%%%%%%%%%

\end{document}
